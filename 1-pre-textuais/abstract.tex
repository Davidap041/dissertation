This work brings with it the main approaches for the kinematic modeling of an industrial robot with 6-\gls{DoF} applying quaternion algebra. For this, it is necessary a whole theoretical foundation behind the spatial representations of position and orientation of a rigid body, as well as the definition of coordinate systems. Linked to this is the theoretical foundation of the classical kinematic approach, evolving from an analytical concept to a numerical one. Approaches to singularities are considered in view of the fragility of the Euler angle model and transformation matrix for the realization of kinematic movement. In view of this, simple quaternions are applied as a contour method for inflection points. In view of the entire theoretical framework developed, the simulated experiment is carried out taking into account the industrial robot UR5 from \textit{Universal Robotic} emulated from the CoppeliaSim EDU environment, which in turn is integrated with the numerical calculation software of high performance Matlab. From differential inverse kinematics in conjunction with quaternion applications, one can experiment with path-following control under disturbance conditions.

% Separe as Keywords por ponto e vírgula.
\keywords{Robotics; Robotic Manipulator; Inverse differential kinematics; Quaternions.}