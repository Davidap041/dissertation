\chapter{Conclusão}

O método de modelagem e controle proposto foi realizado no problema de uma planta industrial envolvendo a sincronia e formação de um grupo de autômatos ao longo de trajetórias pré definidas, em que a partir de alguns eventos modelados pela rede de Petri foi alterado a formação do grupo de autômatos assim como os pontos de sincronia.
O método se apresentou como uma técnica viável e eficiente, pois para aplicações de sistemas com muitos agentes o controle por consenso se apresenta como uma implementação simples sem grande uso de recurso matemático que fornece a sincronia necessária para aplicação de formação ordenada do grupo de autômatos.
No ponto de vista de robustez e adaptabilidade do sistema, observou-se que cada agente respeita as limitações dos agentes vizinhos, seja ela de posição de velocidade, evitando assim colisões e independente da mudança da dinâmica de um agente todo o sistema tem sua dinâmica adaptada, trazendo assim uma sincronia entre os diferentes agentes com diferentes dinâmicas ao longo do sistema.
A principal contribuição desse trabalho é a técnica conjunta de modelagem e controle que  diminui o processamento local em cada agente deixando assim as lógicas de processamento centralizada em um sistema supervisório modelado via rede de petri, além de uma lógica de controle de baixo custo computacional, todavia é necessário uma ótima comunicação entre os agentes,pois a base do controle é dada pela sincronia entre os estados do agente vizinho.
