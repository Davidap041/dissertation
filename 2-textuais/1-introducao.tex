\chapter{Introdução}
\label{cap:introducao}

%Para começar a usar este \textit{template}, na plataforma \textit{ShareLatex}, vá nas opções (três barras vermelhas horizontais) no canto esquerdo superior da tela e clique em "Copiar Projeto" e dê um novo nome para o projeto. 



%Testando o símbolo $\symE$

%\lipsum[5]  % Simulador de texto, ou seja, é um gerador de lero-lero.

%	\begin{alineas}
%		\item Lorem ipsum dolor sit amet, consectetur adipiscing elit. Nunc dictum sed tortor nec viverra.
%		\item Praesent vitae nulla varius, pulvinar quam at, dapibus nisi. Aenean in commodo tellus. Mauris molestie est sed justo malesuada, quis feugiat tellus venenatis.
%		\item Praesent quis erat eleifend, lacinia turpis in, tristique tellus. Nunc dictum sed tortor nec viverra.
%		\item Mauris facilisis odio eu ornare tempor. Nunc dictum sed tortor nec viverra.
%		\item Curabitur convallis odio at eros consequat pretium.
%	\end{alineas}



Um sistema industrial é composto por sensores, atuadores, sinalizadores, controladores entre outros componentes voltados para a realização de determinada cadeia de processos dentro de uma linha de produção. Tal que para realizar determinado processo é necessário uma sincronia entre diversos equipamentos, sensores e atuadores ao longo da planta industrial. Além do desafio de sincronizar uma gama de processos, os sistemas modernos possuem a necessidade de adaptar-se a novas variações e configurações, abrindo espaço para máquinas e sistemas com programação mais robusta e reconfigurável. 
Dado este desafio, as redes de Petri coloridas se oferecem como uma ótima ferramenta de modelagem para os sistemas modernos de manufatura em linha de produção em que há um aumento da versatilidade e flexibilidade da estrutura e também a necessidade de uma programação com alto nível de abstração. 
\cite{WENZELBURGER2019492}

De acordo com
%\cite{discrete}
, as redes de Petri têm sido consideradas com um modelo adequado para um controle supervisório com o objetivo de abranger uma grande classe de problemas e explorar a análise algébrica necessária para otimização. Tratando-se também da análise para a planta não alcançar determinadas marcações indesejadas;

As redes de Petri também são uma ferramenta de modelagem inicial para o algoritmo de programação com ferramentas intrínsecas que analisam o algoritmo para evitar que o sistema entre em exceções,
%\cite{embeddedOO}

% An industrial system is composed of sensors, actuators, signals, controllers and other components dedicated to the implementation of a certain process in a production line. So that to achieve a certain process, there is a need for synchronization between different equipment throughout the plant.

A planta industrial escolhida para este trabalho é relacionada a um processo de montagem, que é composto por um sistema de três agentes, que são duas esteiras e um robô. Uma esteira recebe a parte superior da peça (tampa) e a segunda recebe a parte inferior da peça (base), para posicionar as peças em um local determinado da esteira tem-se o prendedor, uma estrutura metálica que prende a peça à borda da esteira, tal prendedor possui dois tipos de movimento, o de prender e de soltar a peça. Para movimentar a tampa e montá-la em cima da base utiliza-se  o terceiro agente, um braço robótico cartesiano, que possui um movimento no eixo X, ortogonal às esteiras e um movimento no eixo Z, que sobe e desce para levantar e baixar a peça da tampa, respectivamente, através de um atuador pneumático que prende a peça à ponta do braço robótico. O algoritmo de sincronia dos agentes e componentes do processo, tais como atuadores, manipuladores, esteiras e sensores foi modelado por redes de Petri coloridas.

%The industrial plant chosen for this work is related to an assembly process, where it has two conveyors, one that receives the upper part of the object (lid) and the second receives the lower part of the object (base), to position the parts in a determined location of the conveyor there is the clamp, a metal structure that holds the part to the edge of the conveyor, such clamp has two types of movement, to hold and to release the parts. To move the lid and mount it on top of the base, a Cartesian robotic arm is used, which moves on the X axis, orthogonal to the conveyors, and moves on the Z axis, which goes up and down to raise and lower the lid part respectively. .

Para modelagem e controle desse sistema composto por vários agentes, escolheu-se a abordagem por redes de Petri coloridas. As redes de Petri  coloridas são uma ferramenta gráfica e matemática que se adaptam bem a um grande numero de aplicações, tais como protocolos de comunicação, controle de oficinas de fabricação. 
A complexidade dos sistemas, em particular o de fabricação automatizada, leva a uma decomposição de vários níveis de controle, tais como planejamento, escalonamento, coordenação global, coordenação de sub-sistemas e controle direto (autômatos programáveis conectados aos sensores e aos atuadores). %\cite{vallete}
% For modeling and control of this system composed of several agents, the Petri net approach was chosen. The Petri net is a graphical and mathematical tool that adapts well to a large number of applications, such as communication protocols, control of manufacturing workshops. The complexity of systems, particularly automated manufacturing, leads to a decomposition of several levels of control, such as planning, scheduling, global coordination, subsystem coordination and direct control (programmable automata connected to sensors and actuators).

Posteriormente à modelagem por redes de Petri, tal rede será programada utilizando linguagem de programação de auto nível através do paradigma de orientação a objeto, facilitando assim a implementação em sistemas reais, comunicação em auto nível entre o sistema e outros elementos da industria, como clps, supervisórios, sistemas web e possibilitando maior flexibilidade na programação, manutenção e simplificação do código além das ferramentas de análise do modelo a partir da análise da rede de Petri correspondente.

Um sistema multiagente é um sistema que possui mais de um agente, representado por uma entidade independente das outras entidades. Tais entidades se comunicam para a sincronia e execução de um determinado objetivo. No sistema de montagem de peça, considera-se uma entidade como um mecanismo robótico, e outras duas entidades como as esteiras industrias, de modo que o robô é uma entidade independente, que não possui o controle e funcionamento dependente das outras entidades. Assim como o robô as esteiras também têm o funcionamento e controle independente entre si. Para alcançar o objetivo comum de montar as peças os agentes se comunicam entre si através de um protocolo que permite informar o estado de cada agente.

Para a modelagem do sistema em redes de Petri são utilizados os lugares, transições e fichas. O lugar pode ser interpretado como uma condição, um estado associado, por exemplo, sensor ligado, eixo em movimento, peça na posição, etc. Já a transição é associada a uma evento que ocorre no sistema, a um acionamento proposto, tal como movimentar a peça, movimentar robô, acionar a esteira, entre outros. Por fim as fichas são uma indicação que a condição associada ao lugar é satisfeita.

Em 
%\cite{design}, 
sistemas de manufatura reorganizáveis são modelados pela linguagem UML, para descrever as reconfigurações do sistema e na segunda fase  os diagramas são transformados em submodelos da rede de Petri e a relação entre os submodelos e subsistemas são sintetizados em um modelo de rede de Petri de auto nível. Os dois métodos podem analisar comportamentos importantes em relação as propriedades do sistema que são vitais para a modelagem prática dos mesmos.

Em 
%\cite{hybridOO} 
é feita a modelagem de um sistema de produção híbrida, baseada em redes de Petri representando as partes discretas do sistema, equações diferencias representando as partes contínuas e paradigma de orientação a objetos para lidar com a complexidade se sistemas reais, em que cada sub rede é relacionada a uma classe modelando o comportamento de cada objeto da classe. Na dinâmica do sistema a marcação representa o presente estado do objeto. Cada grupo de ação é representada por uma classe eque possui um modelo definido pela RP. 

No presente artigo, para a associação entre os elementos básicos de uma rede de Petri (lugar, transição, fichas) e os elementos básicos de uma planta industrial (sensores, atuadores), modelou-se as seguintes associações, todos os sensores e atuadores da planta são representados por lugares na rede de Petri, tal que o mesmo possui uma ou nenhuma ficha representando respectivamente o estado de ligado ou desligado), acionado ou não acionado, e as transições são lógicas de comando que relacionam lugares e memórias no sistema. 


% De outra forma, abordagens mais simplificadas e de fácil ...

% Em se tratando de soluções numéricas ...

% Um grande desafio presente ...

\begin{comment}   

\section{Objetivos}
\subsection{Objetivos gerais}
% O presente trabalho dedica-se ....


\subsection{Objetivos específicos}
% Os objetivos específicos deste trabalho podem ser resumidos em
\begin{itemize}
    \item Modelagem ... ;
    \item Abordagem ...;
    \item Implementação ....; 
\end{itemize}

\end{comment}

\begin{comment}   
\section{Justificativa}
% Uma vez que ....
\end{comment}

\begin{comment}   
 \section{Motivação}
A abordagem .... 
\end{comment}

\begin{comment}   
\section{Produção científica}
Ao longo do desenvolvimento desta dissertação, foram publicados ou submetidos a
congressos ou periódicos os seguintes artigos:
\begin{itemize}
    \item PAIVA, Davi Alexandre et al. A simple procedure for modeling and identification of a test bench 4-DOF manipulator. In: Congresso Brasileiro de Automática-CBA. 2020.
\end{itemize}
\end{comment}

\section{Organização do trabalho}
Para a estruturação do presente trabalho, adota-se a seguinte metodologia de estudo

\begin{enumerate}
\item 
\textbf{Introdução:} Este capítulo contém as premissas básicas de estudo e evolução dos temas recorrentes na área de Controle Multiagentes. Ainda incluem-se os princípios básicos de apresentação do projeto, tais como os objetivos, a justificativa e a motivação do estudo.

\item \textbf{Sistema Multiagentes: } Partindo-se do princípio mais básico relacionado a modelagem de sistemas multiagentes. Assim, definem-se as representações matemáticas e gráficas de um sistema multiagente assim como técnicas de controle Cooperativo. Por fim são repassadas as principais técnicas de modelagem, representação em grafos e controle de sistemas multiagentes. 
    
\item \textbf{Redes de Petri: } 
    
\item \textbf{Simulação:} 
    
\item \textbf{Conclusão:} Por fim, este último capítulo trata das considerações gerais sobre os conceitos apresentados e uma discussão crítica acerca dos resultados de simulação.

\end{enumerate}

	
