\chapter{Redes de Petri}
\section{Modelagem de Sistemas}
Na modelagem de sistemas a eventos discretos tem-se três elementos principais utilizados na modelagem eventos(instantes e mudanças de estados), atividades(evolução do sistema físico entre dois eventos) e processos (sequência de eventos e atividades no sistema). 

Na evolução dos processos no sistema os processos podem ocorrer de forma totalmente independente entre si, enquanto outras atividades necessitam de uma determinada sincronização ou até uma sequência de eventos prévios. Uma forma de diferenciação de interações entre processos é apresentada por \cite{vallete}, como:

\textbf{Cooperação:} Os processos convergem para um objetivo comum, de modo que anteriormente há uma relativa independência antes do ponto de sincronização.

\textbf{Competição:} Os processos necessitam de um dado recurso, caso esse recurso seja abundante para todos os processos, dados processos poderiam ser descritos de forma independente, caso contrário faz-se necessário o compartilhamento de recursos envolvendo uma exclusão mútua a partir de um ponto de sincronização.

\textbf{Pseudo-Paralelismo:} O paralelismo é apenas aparente e os eventos por mais que sejam independentes nunca serão simultâneos pois são acionados por um relógio comum, a exemplo de um sistema operacional que por mais que processe várias tarefas, porém o processador só processa um ciclo de instrução por vez.

\textbf{Paralelismo Verdadeiro:} Os eventos podem ocorrer de forma simultânea, não existindo uma escala de tempo em comum, a exemplo de vários processadores operando tarefas distintas.

\section{Representação em máquina de estados}
Uma das representações mais clássicas para modelagem de sistemas à eventos discretos é a máquina de estados, para o caso de uma número de estados finitos enumera-se os possíveis estados e descreve-se os eventos referente as mudanças de estado, descrevendo-se assim cada estado a a partir do estado anterior.

O modelo matemático para a máquina de estado finita é dada a partir da equação \ref*{eq:finit_state_machine_equation}, em que $E$ é um conjunto finito de estados, dado pelo estado inicial $E_0$, um alfabeto de entrada $A$, e uma função de transição de estados $\theta$, dado por $\theta : E \times A \rightarrow E$, associando cada par de estado-entrada ao próximo estado.

\begin{equation}\label{eq:finit_state_machine_equation}
    M = (E; A; \theta; E_0)
\end{equation}
De acordo com \cite{vallete} este modelo explicita a noção de eventos e parcialmente a de atividade, não explicitando, entretanto, a noção do processo com as evoluções simultâneas de diversos processos paralelos, de modo que uma máquina de estado finita descreve apenas um único processo sequencial.

\subsection{Modelagem de processos sequenciais}
Para a descrição de vários processos sequências, uma das soluções é representar o sistema por um conjunto de máquinas de estados finitos. Quando as máquinas de estados são independentes, esse modelo se aplica sem dificuldade, porém quando existe competição ou cooperação entre os processos faz-se necessário o uso de processos sequenciais comunicantes. A sincronização é descrita através da intervenção na função de transição de estados $\theta$ de uma máquina.

\subsubsection{Representação com refinamentos sucesssivos}
Um dos contrapontos dessa abordagem é que independente do método utilizado, a representação das comunicações entre as máquinas é diferente da representação interna da sequência de uma máquina. Portanto, tal abordagem não é compatível com a abordagem top-down de refinamento sucessivos. É necessário desde do início da modelagem a escolha de uma decomposição que não será colocada em causa a posteriori. 

\subsubsection{Explosão combinatória}
Outro ponto de análise dessa forma de modelagem é que para cada informação partilhada entre as máquina ou troca de sinais entre as máquinas é necessário analisar o comportamento global do sistema através do recálculo de uma nova máquina de estado que descreva o sistema de forma global. Neste caso, ocorre a problemática da explosão combinatória do número de estados definida pela relação entre $k$ máquinas e $n$ estados, produzindo uma máquina de $n^k $ estados, ocorrendo uma explosão combinatória a medida que k e n aumentam.

\subsubsection{Não-independência de submáquinas e bloqueio}


\section{Modelagem utilizando Rede de Petri}

A rede de Petri é uma ferramenta gráfica e matemática para modelagem e controle de sistemas à eventos discretos, dado que o sistema a ser escolhido é um sistema que pode ser modelado através de tal ferramenta, com o intuito de obter uma visualização gráfica do processo, implementar lógica de controle e sincronismo, analisar propriedades da rede entre outras.

A descrição dos eventos e transições do sistema é dada através da rede de petri pelos lugares, fichas e transições, para representar os "estados" do sistema são utilizados os lugares, já as transições movimentam os recursos, ou seja, as fichas de um lugar para outro, dada a condição que a transição só possa ser disparada caso os lugares a ela ligada estejam completas com os recursos requisitados, fazendo assim interdependências em que um determinado estado só pode ser alcançado caso determinadas condições satisfeitas.

Para modelagem e controle desse sistema composto por vários agentes, escolheu-se a abordagem por redes de Petri coloridas. As redes de Petri  coloridas são uma ferramenta gráfica e matemática que se adaptam bem a um grande numero de aplicações, tais como protocolos de comunicação, controle de oficinas de fabricação. 
