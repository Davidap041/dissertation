\chapter{Sistema Multiagentes}

\section{Teoria dos Grafos}
No estudo da interação e comportamento entre sistemas dinâmicos, as interconexões entre os agentes e o fluxo de informações formam uma rede de comunicação. Essa rede é modelada através da teoria dos grafos em que cada sistema é representado por um nó, também chamado de agente.
% Adicionar Referência do livro que fala sobre

Um grafo é um par $G = (V, E)$, tal que $V = \{v_{1},v_{2}, ...,v_{N}\}$ é um conjunto de $N$ nós ou vértices e $E$ um conjunto de vetores ou arcos. Um elemento pertencente a $E$ é um par $(v_{i}, v_{j})$ tal que é um vetor que liga $v_{i}$ à $v_{j}$, e é representado como uma flecha em que a cauda está em $v_{i}$ e a ponta em $v_{j}$ como demonstrado na figura. 
% Adicionar Grafo exemplo desenhado : exemplo de nós e arcos em um grafo

Os graus de liberdade de entrada de um dado nó $v_i$ é definido como o número de vetores que a ponta da flecha se encontra em $v_i$. Do mesmo modo, os graus de liberdade de saída de um nó $v_i$ é dado pelo número de vetores que em $v_i$ se encontra a cauda da flecha.

Associado à cada aresta de um elemento em $E = (v_i, v_j)$ tem-se um peso $a_{ij} > 0$. O peso $a_{ij}$ representa a força de interação entre os nós $v_i$ e $v_j$. De modo que quanto maior o peso maior a influência tem o comportamento do agente $j$ sobre o agente $i$.
    %revisar esse parágrafo
Um grafo é dito bidirecional se e somente se $a_{ij} \neq 0$ e $a_{ji} \neq 0$, então tem-se que a comunicação entre agentes flui bidirecionalmente. Um grafo é dito unidirecional se $a_{ij} = a_{ji}$, para qualquer $i$ e $j$.


\section{Teoria algébrica dos grafos e consenso do controle cooperativo}
\label{sec:posi_ori}

